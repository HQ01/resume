\documentclass{onkursen-resume}

\usepackage{fancyhdr}
\pagestyle{fancy}
\lhead{51 Dudley Ln \#120, Stanford, CA 94305\\ \noindent (918) 407$-$2319}
\rhead{onkursen@gmail.com\\ \noindent github.com/onkursen}
\chead{{\bf \Huge Onkur Sen}}
\lfoot{}
\cfoot{}
\rfoot{}

\begin{document}

\vspace{0.5mm}

\begin{multicols}{2}

\section*{Education}

\noindent {\bf \large Stanford University}\\
September 2014 | June 2016 (expected)\\
M.S. Computer Science\\
Dual Depth: AI \& HCI\\

\vspace{6mm}

\noindent {\bf \large Rice University}\\
August 2010 | May 2013\\
B.S. Physics, B.A. Mathematics\\
Minor: Computational and Applied Mathematics

\end{multicols}

\vspace{-1mm}

\hr

\section*{Software Engineering Experience}

\affiliation{Palantir}{Forward Deployed Engineer (Philanthropy Team)}{Java}{May 2013 | September 2014} 
\begin{itemize*}
\item Developed custom dashboard visualizations of commitments by the Clinton Global Initiative (CGI) over the past ten years
\item Maintained and upgraded infrastructure for the National Center for Missing and Exploited Children (NCMEC)
\item Implemented secure password hashing and executed cloud migration for START/UMD Global Terrorism Database
\end{itemize*}

\affiliation{Plum District}{KPCB Engineering Fellow}{Ruby/Rails}{May 2012 | August 2012}
\begin{itemize*}
\item Implemented tracking mechanism for Remarketing, Omniture, and Google Analytics
\item Corrected redemption of vouchers and allowed view of past offers in business center
\end{itemize*}

\affiliation{TripAdvisor}{Software Engineer Intern}{Java/Velocity}{December 2011 | January 2012}
\begin{itemize*}
\item Removed cross-site scripting (XSS) vulnerabilities and improved display of Facebook likes, ratings, and recommendations
\end{itemize*}

\hr

\section*{Research}

\affiliation{Apparition}{Advisor: Dr. Michael Bernstein, Stanford University}{JavaScript (Meteor)}{Spring 2014 | Present}
\begin{itemize*}
\item Developing webapp that utilizes crowdsourcing around a Method Draw canvas to quickly prototype interfaces
\end{itemize*}

\affiliation{Searching for Supersymmetric Top Quarks at the LHC}{Advisor: Dr. Paul Padley, Rice University}{Python}{Fall 2012 | Spring 2013}
\begin{itemize*}
\item Used boosted decision trees in ROOT TMVA to isolate decay of stop quarks from background top-top interactions
\item Extended on phenomenological data and theory from Bhaskar Dutta et al. (Texas A\&M)
\item {\bf Publication:} Sen, O. and Padley, B.P. Searching for Supersymmetric Top Quarks at the LHC [Thesis]. April 22, 2013.
\end{itemize*}

\affiliation{Melody Analysis and Harmony Generation}{Advisor: Dr. Kurt Stallmann, Rice University}{Python}{Fall 2011 | Fall 2012}

\begin{itemize*}
\item Determined key of input score given only melodic line and generated complementary harmonic progression
\item {\bf Publication:} Sen, O. and Stallmann, K. Analysis of Melody Through Key Definition and Generation of Complementary Harmonies. Rice Undergraduate Research Symposium. Houston, TX, April 13, 2012.
\end{itemize*}

\affiliation{Computationally Generating Musical Variations}{Advisor: Dr. Sandip Sen, University of Tulsa}{Java}{Fall 2009 | Fall 2011}

\begin{itemize*}
\item Created systematic framework for representing musical scores and used genetic algorithms to create variations on themes
\item {\bf Publication:} Sen, O. Creating Musical Variations Using Genetic Algorithms. {\em American Junior Academy of Sciences}. Washington, DC, February 16$-$20, 2011.
\end{itemize*}

\affiliation{Social Networks and Norm Emergence}{Advisor: Dr. Sandip Sen, University of Tulsa}{Java}{Fall 2008 | Summer 2009}

\begin{itemize*}
\item Analyzed comparative speed of emergence of a norm in social networks with different topologies and behavioral patterns
\item {\bf Publication:} Sen, O. and Sen, S. Effects of Social Network Topology and Options on Norm Emergence. {\em Lecture Notes in Artificial Intelligence} Vol. 6069, p. 211$-$222, Springer-Verlag, 2010.
\end{itemize*}
 
\affiliation{Social Dilemmas and Aspiration Levels}{Advisor: Dr. Sandip Sen, University of Tulsa}{Java}{Fall 2007 | Summer 2009}

\begin{itemize*}
\item Developed algorithmic approach to solve the Tragedy of the Commons in a multi-agent system using aspiration levels
\item {\bf Publication:} Sen, O. and Sen, S. Solving the Tragedy of the Commons by Adapting Aspiration Levels. {\em Proceedings of COIN@IJCAI09}. San Diego, CA, July 11, 2009.
\end{itemize*}

\hr

\section*{Projects}

\noindent
\begin{tabular}{lrlr}
\project{Contagion}{model of diseases spreading across social networks (Hack Week)}{}{Python, D3}{2013}
\project{Rice University Catalyst}{website}{catalyst.rice.edu}{HTML/CSS}{2011|2013}
\project{Rice University South Asian Society}{website}{sas.rice.edu}{HTML/CSS}{2011|2013}
%\project{Sangleet}{wrote/choreographed/directed a 15-minute musical}{bit.ly/sangleet}{}{2011}
\end{tabular}

\end{document}