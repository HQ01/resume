\documentclass{onkursen-resume}

\usepackage{fancyhdr}
\pagestyle{fancy}
\lhead{657 Everett Ave \#4, Palo Alto, CA 94301\\(918) 407$-$2319}
\rhead{onkursen@gmail.com\\github.com/onkursen}
\chead{{\bf \Huge Onkur Sen}}
\lfoot{}
\cfoot{}
\rfoot{}

\begin{document}

\begin{multicols}{2}
\section*{Education}
\noindent {\bf \large Rice University}, August 2010 | May 2013\\
Bachelors of Science in Mathematics and Physics\\
Minor in Computational and Applied Mathematics\\
GPA: 3.55/4.00

\section*{Academic Honors}
\noindent
\honor{Rice Trustee Distinguished Scholar}{August 2010}\\
\honor{Rice Century Scholar}{August 2010}\\
\honor{Robert C. Byrd Scholar}{May 2010}\\
\honor{National Merit Scholar}{March 2010}
\end{multicols}

\hr

\section*{Software Engineering Experience}

\job{Palantir}{Forward Deployed Philanthropy Engineer}{Python, Java}{May 2013 | Present} 
\begin{itemize*}
\item Forward deployed engineer with philanthropy team
\item Integrated parcel data on sites affected by May 2013 Oklahoma City tornado to aid disaster relief efforts by Team Rubicon
\item Developing global/public health instance combining open public data, custom metrics, and innovative visualizations
\end{itemize*}

\job{Plum District}{KPCB Engineering Fellow}{Ruby/Rails}{May 2012 | August 2012}
\begin{itemize*}
\item Implemented tracking mechanism for Remarketing, Omniture, and Google Analytics
\item Fixed view of past offers in business center and corrected redemption of vouchers and deals
\end{itemize*}

\job{TripAdvisor}{Software Engineer Intern}{Java/Velocity}{December 2011 | January 2012}
\begin{itemize*}
\item Removed cross-site scripting (XSS) vulnerabilities and implemented JavaScript escaping for text
\item Improved relevance of display of Facebook likes, ratings, and recommendations
\end{itemize*}

\hr

\section*{Research}

\research{Searching for Supersymmetric Top Quarks at the LHC}{Python}{Fall 2012 | present}
\begin{itemize*}
\item Using boosted decision trees in ROOT TMVA to isolate decay of squarks from background top-top interactions
\item Extending on phenomenological data and theory from Bhaskar Dutta et. al. (Texas A\&M)
\end{itemize*}

\research{Melody Analysis and Harmony Generation}{Python}{Fall 2011 | Fall 2012}

\begin{itemize*}
\item Modeled common practice music theory with respect to key structures
\item Determined key of input scores given only melodic line; also generated complementary harmonic progression
\end{itemize*}
\vspace{-2mm}
$\aItem$ {\bf Preprint:} Sen, O. and Stallmann, K. Analysis of Melody Through Key Definition and Generation of Complementary $\aSpace$ Harmonies. Rice Undergraduate Research Symposium. Houston, TX, April 13, 2012.
\vspace{2mm}

\research{Computationally Generating Musical Variations}{Java}{Fall 2009 | Fall 2011}

\begin{itemize*}
\item Created systematic framework for representing musical scores
\item Used genetic algorithms to create variations on themes
\end{itemize*}
\publication{Sen, O. Creating Musical Variations Using Genetic Algorithms. {\em American Junior Academy of Sciences}. $\aSpace$ Washington, DC, February 16$-$20, 2011.}

\research{Social Networks and Norm Emergence}{Java}{Fall 2008 | Summer 2009}

\begin{itemize*}
\item Built social networks with different topologies and behavioral patterns
\item Analyzed comparative speed of emergence of a norm in the networks
\end{itemize*}
\publication{Sen, O. and Sen, S. Effects of Social Network Topology and Options on Norm Emergence.
 {\it Lecture Notes} $\aSpace$ {\it in Artificial Intelligence} Vol. 6069, p. 211$-$222, Springer-Verlag, 2010.}

\research{Social Dilemmas and Aspiration Levels}{Java}{Fall 2007 | Summer 2009}

\begin{itemize*}
\item Developed algorithmic approach to solve the Tragedy of the Commons in a multi-agent system using aspiration levels
\item Formulated mathematical model placing an upper bound on convergence time
\end{itemize*}
\publication{Sen, O. and Sen, S. Solving the Tragedy of the Commons by Adapting Aspiration Levels. {\em Proceedings of} $\aSpace$ {\it  COIN@IJCAI09}. San Diego, CA, July 11, 2009.}

\hr

\section*{Projects}

\noindent
\begin{tabular}{llrr}
\project{Rice University Catalyst}{website}{catalyst.rice.edu}{HTML/CSS}{2011|2013}
\project{Rice University South Asian Society}{website}{sas.rice.edu}{HTML/CSS}{2011|2013}
%\project{Simple Charts}{real-time updating charts}{simplecharts.heroku.com}{D3.js/NVD3.js}{2012}
%\project{vote.me}{voting app for distributed elections}{voteme.heroku.com}{Ruby/Rails}{2012}
\project{Sangleet}{wrote/choreographed/directed a 15-minute musical}{bit.ly/sangleet}{}{2011}\\
\end{tabular}

\hr

\section*{Activities}

\begin{description*}
\activity{Rice University South Asian Society}{co-president (2012|2013), treasurer (2011|2012)}
\activity{Rice University Catalyst}{co-editor-in-chief (2012|2013), executive editor (2011|2012)}
\activity{Will Rice College}{academic fellow (2012|2013)}
\activity{Partnership for Advancement \& Immersion of Refugees}{volunteer/photographer (2010|2011)}
\end{description*}

\end{document}